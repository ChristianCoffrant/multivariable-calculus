
\chapter{Motion}

This chapter covers the following ideas. 

% A list of objectives for the chapter
%\begin{enumerate}
%\item ...
%\end{enumerate}


\begin{enumerate}
\item Describe projectile motion.  Develop formulas which are valid if
  we neglect air resistance and consider only acceleration due to
  gravity.  Use your model to develop formulas for the range, maximum
  height, and flight time.
\item Develop the $TNB$ frame for describing motion. Add to your model
  the concepts of curvature, osculating circle, torsion, and the
  tangential and normal components of acceleration. Be able to prove
  the relationships that you develop in the $TNB$ frame.
\end{enumerate}


%%% Local Variables: 
%%% mode: latex
%%% TeX-master: "../multivariable-calculus"
%%% End: 
%$

\section{Projectile Motion}

The main point of this section is not for you to memorize the formulas
and put numbers in them, but rather that you can actually derive these
formulas.  The homework in the textbook provides you with a bunch of
problems where you are just putting in numbers and solving. I suggest
that you start each problem from scratch and derive the formulas (as
this is what I will test you on).

\note{make picture and table of the quantities}

If an object has initial velocity {$ \vec v_0 $}, initial position {$
  \vec r_0 $}, and acceleration {$ \vec a(t) $}, then you can find the
position at any given time by integrating, since $\vec v(t)=\int \vec
a(t) dt$, and $\vec r(t)=\int \vec v(t)dt$.  Projectile motion describes
the ideal path of motion of an object which is fired into the air at a
given angle, $\alpha$, and a given speed, $v_0$, assuming that gravity
$\vec a(t) =\langle0,-g\rangle$ is the only force which acts on the
projectile. Note that our coordinate system has the gravity vector
pointing downwards. Notationally, we let $x(t)$ represent the horizontal position at time $t$, $y(t)$ represent the vertical position at time $t$, $\vec r(0)=\vec r_0 = \langle
x_0,y_0\rangle$, $v_{x_0} = v_0\cos \alpha$, $v_{y_0}=v_0\sin \alpha$, and
$\vec{v_0}=\langle v_{x_0},v_{y_0}\rangle$. Integration gives $\vec v(t)=\vec a
t+\vec v_0$, and $\vec r(t) =\frac{1}{2}\vec a t^2 +\vec v_0 t +\vec
r_0$. Therefore, we have $x(t) = v_{x_0}t +x_0= v_0\cos\alpha+x_0$ and
$y(t) = -\frac{1}{2}gt^2+v_{y_0}t+y_0=-\frac{1}{2}gt^2+v_0\sin\alpha
t+y_0$. To simplify the calculations, most of the time the coordinate
axis is placed with the origin at $(x_0,y_0)$, which simplifies the
formulas to be $x(t) = v_{x_0}t$ and $y(t) =
-\frac{1}{2}gt^2+v_{y_0}t$.

Using this formula, we can compute the height, flight time, and range
by using tools from first-semester calculus. The maximum height is
achieved when the velocity is $0$ in the $y$ direction.  Hence we want
to solve $0=-gt+v_{y_0}$ or $t=v_{y_0}/g = v_0\sin\alpha/g$.  The max
height is the $y$ value at this value of $t$, so the max height is
$y_{\text{max}} = -\frac{1}{2}g(v_{y_0}/g)^2+v_{y_0}v_{y_0}/g =
\frac{1}{2}v_{y_0}^2/g$.  Since the motion follows a parabola, the
range is found by calculating the $x$ value at twice this time, or $R
= v_{x_0}(2v_{y_0}/g) = 2v_0^2\sin\alpha\cos\alpha/g=\sin(2\alpha)/g$.  We could also
find the range by solving $y(t)=y_0$ and plugging the corresponding
$t$ value into $x(t)$.


Here is an example.  An archer stands 6ft above ground level and
shoots an arrow at an object which is 90 feet away in the horizontal
direction and 74 ft above ground. The archer needs the arrow to hit
the target at the peak of its parabolic path. For the purposes of this
example, let $g = 32 \text{ft}/\text{s}^2$. (see the textbook, page
908) What initial velocity and firing angle are needed to achieve this
result? To answer this, we first decide where to place the origin.  We
will place the origin at 6ft above ground, so that the max height is
68 ft and it is achieved 90 ft away horizontally.  We need to solve
$\vec r(t_m)=\langle90,68\rangle = \langle v_{x_0}t_m, -\frac{1}{2}gt_m^2+v_{y_0}t\rangle$,
and $\vec v(t_m)= \langle v_{x_0},0 \rangle=\langle v_{x_0},-gt_m+v_{y_0}\rangle $ for $t_m,
v_{x_0}, v_{y_0}$, as then $v_0=\sqrt{ v_{x_0}^2+ v_{x_0}^2}$ and $\alpha =
\arctan( v_{y_0}/ v_{x_0})$.  We have $ 0 = -gt_m+v_{y_0}$ or $t_m =
v_{y_0}/g$.  The $y$ coordinate of the position gives
$68=-\frac{1}{2}g(v_{y_0}/g)^2+v_{y_0}v_{y_0}/g =
\frac{1}{2}v_{y_0}^2/g$. Hence $v_{y_0} = \sqrt{2\cdot 68\cdot g}$. The $x$
coordinate of the position gives $90 = v_{x_0}v_{y_0}/g$, or $v_{x_0}
= 90g/v_{y_0}=90g/\sqrt{2\cdot 68\cdot g}$. The rest of the variables can now
be solved for to find the initial firing angle.




\section{Arc Length}

Recall the formulas for arc length are, depending on how the function
is defined, $ s=\int_a^b \sqrt{[\frac{dy}{dx}]^2+1}\;dx$, $s=\int_c^d
\sqrt{1+[\frac{dx}{dy}]^2}\;dy$, or $s=\int_a^b
\sqrt{[\frac{dx}{dt}]^2+[\frac{dy}{dt}]^2}\;dt $.  Using differential
notation {$ ds=\sqrt{dx^2+dy^2} $}, these formulas can all be
summarized by the formula $ s=\int_C ds $, where $C$ represents the curve
over which we are integrating. This notation introduces the notation
for line integrals. By parametrizing the curve $C$ as {$ r(t)=\langle
  x(t),y(t)\rangle $}, we see $ s=\int_C ds = \int_a^b |r^\prime(t)| dt $.  In other
words, a little change of arc length {$ ds $} is equal to the product
of speed $|\vec r^\prime(t)|$ and a little change in time $dt$.  This is
the same formula learned in grade school: distance = rate $\times$ time,
where now we are just adding up a bunch of distances using definite
integrals.

\subsection{Reparametrizing by arc length}
From now on we will assume that curves are smooth, which means that
the curve is differentiable and $\vec r^\prime(t)\neq \vec 0$ (the velocity is
never zero).  When we follow a space curve $\vec r(t)$, the speed
$|\vec r^\prime(t)|$ traveled depends on the parameter {$t$}. At some
points along the curve the speed could be larger than at others.  We
can introduce a new parametrization by speeding up if the the speed is
less than one and slowing down if the speed is greater than one. This
new parametrization will move at constant speed 1, so that every one
unit increase in time results in a one unit increase in length. For a
curve $\vec r(t)$, $a\leq t\leq b$, let $s(t)=\int_a^t|\vec r^\prime(\tau)|\;d\tau$ (note
that we use {$ \tau $} as a dummy variable since $t$ is already used in
the bounds of the integral). We see that $s(t)$ calculates the length
traveled by the curve between $a$ and $t$.  The fundamental theorem of
calculus shows that $\frac{ds}{dt} = |\vec r^\prime(t)|$. Since the speed
is never zero, we can find an inverse function $t(s)$, which, given an
arclength $s$, tells us the amount of time needed to travel the
distance $s$. The inverse function theorem states that $\frac{dt}{ds}
= \frac{1}{ds/dt} = \frac{1}{| \vec r^\prime(t)|}$. The chain rule then
gives the derivative of $\vec r(t(s))$ with respect to the arc length
parameter $s$ as $\frac{dr}{ds} = \frac{d\vec r}{dt}\frac{dt}{ds} =
\frac{\vec r^\prime}{|\vec r^\prime(t)|}$, which is a unit vector. Hence if we
use $s$ as the parameter, we traverse the curve at constant speed $1$.
Theoretically it is possible to always reparametrize any smooth curve.
However, in practice it may not always be easy to actually find the
parametrization.

For the helix $ \vec r(t)=\langle\cos(t),\sin(t),t\rangle $, we can compute the
following: $r^\prime(t) =\langle-\sin(t),\cos(t),1\rangle$, $|r^\prime(t)| = \sqrt{(-\sin
t)^2+(\cos t)^2+1^2}=\sqrt{2}$. The length of one coil is $s =
\int_0^{2\pi}\sqrt{2}dt = 2\pi\sqrt 2$. The arc length parameter is $s(t) =
\int_0^{t}\sqrt{2}dt = t\sqrt{2}$.  Hence $t=s/\sqrt{2}$.  So if we
reparametrize the curve using the composite $\vec r(t(s)) = 
\langle\cos(\frac{s}{\sqrt 2}),\sin(\frac{s}{\sqrt 2}),\frac{s}{\sqrt 2}\rangle $,
we find $\frac{d\vec r}{ds} = \langle-\frac{1}{\sqrt{2}}\sin(\frac{s}{\sqrt
2}),\frac{1}{\sqrt{2}}\cos(\frac{s}{\sqrt 2}),\frac{1}{\sqrt 2}\rangle $ and
$\left|\frac{d\vec r}{ds}\right|=1$. 



\section{The $TNB$ frame}

For a space curve $\vec r(t) = \langle x(t),y(t),z(t)\rangle$, the $TNB$ frame is
an orthogonal collection of unit vectors which describe the
tangential, normal, and binormal (tangential cross normal) directions
of motion. Such a frame is necessary for a stationary observer to
understand how the world looks to a moving object.  The $TNB$ frame
gives an observer a way to place points in an $xyz$ coordinate frame.
Imagine two friends, one on the ground and another on a spacecraft
which moves in the tangential direction ({$ \vec T $}) with the left
wing always pointing in the direction $\vec N$ of acceleration which
is orthogonal to the tangential direction of motion. Then the binormal
direction ({$ \vec B $}) is the direction the head of the person on a
spacecraft would point. The $TNB$ frames gives a way of allowing the
observer on the ground to give directions to the person in the
spacecraft. The following table summarizes the discussion which
follows.

 
\begin{center}
\begin{tabular}{|c|c|c|}
\hline
Unit Tangent Vector & $\vec T$ & $\frac{d\vec r}{ds} = \frac{d\vec
r/dt}{ds/dt} = \frac{\vec r^\prime(t)}{|\vec r^\prime(t)|}$\\\hline
Curvature Vector & $\vec \kappa $& $\frac{d\vec T}{ds} =\frac{d\vec
T/dt}{ds/dt} = \frac{d\vec T/dt}{|\vec v|} = \frac{\vec T^\prime(t)}{|\vec
r^\prime(t)|} $\\\hline
Curvature (not a vector, but a scalar)& $ \kappa $&$= \left|\frac{d\vec
T}{ds}\right| =\left|\frac{d\vec T/dt}{ds/dt}\right| =
\frac{\left|d\vec T/dt\right|}{|\vec v|}= \frac{|\vec T^\prime(t)|}{|\vec
r^\prime(t)|}  $ \\\hline
Principal unit normal vector & $ \vec N$& $ \frac{1}{\kappa}\frac{d\vec
T}{ds} = \frac{\vec \kappa }{|\vec \kappa |} = \frac{d\vec T/dt}{|d\vec T/dt|} = \frac{\vec T^\prime(t)}{|\vec
T^\prime(t)|}$\\\hline
Radius of curvature & $ \rho$ & $1/\kappa$\\\hline
Center of curvature &  & $\vec r(P)+\rho(P)\vec N(P)$ \\\hline
Binormal vector & $ \vec B$& $ \vec T\times\vec N$\\\hline
Torsion & $ \vec \tau $ & $ -\frac{d\vec B}{ds}\cdot \vec N = -\frac{d\vec
B/dt}{ds/dt}\cdot \vec N = -\frac{\vec B^\prime(t)}{|\vec r^\prime(t)|}\cdot \vec N
$\\\hline
Tangential Component of acceleration & $ a_T$ & $ \vec a \cdot \vec T =
\frac{d}{dt}|\vec v|$\\\hline
Normal Component of acceleration & $ a_N$ & $ \vec a \cdot \vec N = \kappa
\left(\frac{ds}{dt}\right)^2 = \kappa |\vec v|^2$\\\hline
\end{tabular}
\end{center}
 
 

\subsection{The unit tangent vector $\vec T$}
The velocity {$\vec r^\prime(t) = \vec v(t)$} gives us the tangential
direction of motion. Division by the magnitude gives the unit vector
(called the unit tangent vector) {$ \vec T  = \frac{\vec r^\prime(t)}{|\vec
r^\prime(t)|} = \frac{\vec v}{|\vec v|}$}
Alternatively, if the curve is parametrized by arc length, then the
speed is already 1, so the unit tangent vector is also $ \vec T  =
\frac{d\vec r}{ds}  =  \frac{d\vec r}{dt} \frac{dt}{ds} = \vec r^\prime(t)
\frac{1}{ds/dt}  = \vec r^\prime(t) \frac{1}{|\vec r^\prime(t)|}= \frac{\vec
v}{|\vec v|}$.

\subsection{If a vector valued function has constant length, then its
derivative is orthogonal to the function.}
First note that the product rule works for vector valued functions
when considering the dot or cross product of two space curves (in
general mathematicians define operations as products if those
operations obey the product rule for derivatives). If the length of a
vector is always constant, i.e. {$ |\vec r(t)|=c $}, then we have {$
  |r(t)|^2 = \vec r(t)\cdot \vec r(t) = c^2 $}.  Taking derivatives of
both sides (and using the product rule) gives {$ \vec r(t)\cdot \vec
  r^\prime(t)+\vec r^\prime(t)\cdot \vec r(t)=0 $}, or {$2\vec r(t)\cdot \vec r^\prime(t)=0
  $}.  This means that {$ \vec r(t)\cdot \vec r^\prime(t)=0 $}, or that {$ \vec
  r(t) $} and {$ \vec r^\prime(t) $} are orthogonal. So if a vector valued
function has constant length, then its derivative is orthogonal to the
curve itself.

\subsection{Curvature $\kappa$ and Principal Unit Normal Vector $\vec
N$}
The unit tangent vector {$ \vec T $} gives us the tangential direction
of motion. The derivative of the unit tangential vector tells us how
the direction of motion is changing. Since the unit tangent vector
always has length one, its derivative is perpendicular to the tangential
direction.  Curvature is a measure of the rate of change of the unit
tangent vector per unit length. The curvature vector {$ \vec \kappa =
  \frac{d\vec T}{ds} $} points in a direction normal to {$ \vec T $}.
The magnitude of the curvature vector is called the curvature, and
written {$ \kappa = |\frac{d\vec T}{ds}| $}. The unit vector in the
direction of the curvature vector is called the principal unit normal
vector, and written {$ \vec N $}. As reparametrizing by arc length can
be difficult, it is convenient to give formulas for curvature that can
be computed from a given parametrization. To do this, we apply the
inverse function theorem again to see that $dt/ds = 1/(ds/dt)=1/|\vec
r'(t)|$, so $\kappa=|\frac {d\vec T}{ds}| = |\frac {d\vec T}{dt} \frac
{dt}{ds}|$, which then becomes $\kappa=|\vec T'(t)|/|\vec r'(t)|$.  These
formulas are also listed at the beginning of this section.  

For example, a circle of radius $\alpha$ has $\vec r(t)=\langle\alpha\cos t, \alpha \sin
t\rangle$, so $\vec T(t)=\vec r'(t)/|\vec r'(t)| = \langle-\alpha \sin t, \alpha \cos
t\rangle/\alpha=\langle-\sin t, \cos t\rangle$, so $\kappa=|\vec T'(t)|/|\vec r'(t)|=1/\alpha$.
Therefore, the curvature of a circle of radius $\alpha$ is {$\kappa= 1/\alpha $}.

\subsection{Circle of curvature (osculating circle)}
The circle of curvature at a point $P$ on a curve where the curvature
is nonzero is a circle in the plane containing the unit tangent and
principle unit normal vectors which is tangent to the curve at $P$,
has the same curvature as the curve at $P$, and whose center lies in
the direction of the principal unit normal (i.e. the center is $\vec r
+\rho \vec N$). This circle is the best approximating circle to the curve
at $P$. The radius of the circle is {$ \rho=\frac{1}{\kappa} $}.  The center
of the circle of curvature is called the center of curvature.


\subsection{The Binormal vector $\vec B$ and Torsion $\tau$}
The binormal vector is {$ \vec B=\vec T \times \vec N $}. Notice that $\vec
B$ is already a unit vector. The binormal vector provides the {$z$}
axis for describing the world from the viewpoint of an object in
motion where the {$ x $} and {$ y $} axes are given by the {$ \vec T
$} and {$ \vec N $} directions, respectively. 

The derivative $ \frac{d\vec B}{ds}  = \frac{d\vec B /dt}{ds/dt}$
measures how quickly the binormal vector changes as you move along a
curve (or how quickly the object is twisting).  Since {$ \vec B $} is
constant length, the derivative {$ \frac{d\vec B}{ds} $} is orthogonal
to {$ \vec B $}.  The following computations show that {$ \frac{d\vec
B}{ds} $} is also orthogonal to $\vec T$, which means that {$
\frac{d\vec B}{ds} $} must be parallel to $\vec N$. We compute (using
the product rule) {$ \frac{d\vec B}{ds} = \frac{d(\vec T\times \vec N)}{ds}
= \vec T\times \frac{d\vec N}{ds}+ \frac{d\vec T}{ds}\times \vec N = \vec T\times
\frac{d\vec N}{ds}+\vec 0 $}. The last $\vec 0$ comes because {$
\frac{d\vec T}{ds} $} is parallel to {$ \vec N $}, and the cross
product of parallel vectors is the zero vector. We see from this
computation that {$ \frac{d\vec B}{ds} = \vec T\times \frac{d\vec N}{ds}
$}, which means that {$ \frac{d\vec B}{ds} $} is orthogonal to {$ \vec
T $}.  Since {$ \frac{d\vec B}{ds} $} is orthogonal to both {$ \vec B
$} and {$ \vec T $}, it must be a scalar multiple of {$ \vec N $}. 
Torsion {$ \tau $} is the opposite of the scalar component of {$
\frac{d\vec B}{ds} $} in the direction of {$ \vec N $}, i.e. {$
\tau=-\frac{d\vec B}{ds}\cdot \vec N $}. Torsion is a measure of the rate at
which acceleration is causing an object to rotate out of the plane
containing {$ \vec T $} and {$ \vec N $}. Objects which are spiraling
clockwise (as seen from behind the object) around some axis have
positive torsion. Spiraling counterclockwise results in negative
torsion. If you wrap your hand around the $\vec T$ vector in the
direction of {$ \frac{d\vec B}{ds} $}, then a clockwise rotation has
your thumb point in the $\vec T$ direction.  This is the reason for
the choice of sign.

\subsection{Tangential and Normal Components of acceleration}
We now decompose the acceleration into tangential and normal
components. The scalar component of acceleration in the tangential
direction is called the Tangential Component of Acceleration $a_T$,
and the scalar component of acceleration in the normal direction is
called the Normal Component of Acceleration $a_T$. These can be
computed using projections, namely {$ a_T=\text{comp}_{\vec T}\vec a 
= \frac{\vec a \cdot \vec T}{|\vec T|} = \vec a \cdot \vec T $}, and {$
a_N=\text{comp}_{\vec N}\vec a  = \frac{\vec a \cdot \vec N}{|\vec N|} =
\vec a \cdot \vec N $}.
Alternatively, we can write {$ \vec v = \frac{d\vec r}{dt} =
\frac{d\vec r}{ds}\frac{ds}{dt} = \vec T \frac{ds}{dt} $}, and then
compute (using the product rule)
$$ \vec a = \frac{d\vec v}{dt} = \frac{d}{dt}\left( \vec T
\frac{ds}{dt} \right) = \vec T  \frac{d}{dt}\frac{ds}{dt} +
\frac{d}{dt}\vec T  \frac{ds}{dt} = \vec T  \frac{d^2 s}{dt^2} +
\frac{\frac{d}{dt}\vec T}{\frac{ds}{dt}}  \left(\frac{ds}{dt}\right)^2
= \vec T  \frac{d}{dt}|\vec v| + \vec \kappa  \left(\frac{ds}{dt}\right)^2
=   \frac{d}{dt}|\vec v| \vec T + \kappa   \left(\frac{ds}{dt}\right)^2
\vec N.$$
This shows that the acceleration is in the plane formed by the
tangential and principal unit normal vectors. 

There are many other formulas for computing $a_T$ and $a_N$. You don't
need to memorize them, rather you should be able to derive them.
Following are useful formulas for doing basic computations.  Modern
technology makes doing the computations in all cases simple,
regardless of the formula.  You should know how to derive these
formulas given the hints I provide: {$ a_N =\sqrt{ |\vec a|^2-a_T^2} 
$} (this is very useful for computing {$ a_N $}, just draw the vectors
and use the Pythagorean theorem), {$ a_N = \vec a \cdot \vec N $} (the
hard part here is computing {$ \vec N $} ), {$ \kappa = \frac{\vec v \times \vec
a}{|\vec v|^3} $} (write {$ \vec v $} and {$ \vec a $} in terms of {$
\vec T $} and {$ \vec N $}, and then compute the magnitude
symbolically, and solve for {$ \kappa $}).

\subsection{An Example}
We will compute the quantities above for the helix $\vec r(t) = \langle\cos
t,\sin t, t\rangle$.
$\vec v(t) = \vec r^\prime(t) = \langle-\sin t,\cos t, 1\rangle$, and speed $= |\vec
r^\prime(t)| = \sqrt{2}$,
$\vec T(t) = \vec r^\prime(t)/|\vec r^\prime(t)| = \frac{1}{\sqrt{2}}\langle-\sin
t,\cos t, 1\rangle$,
$d\vec T/dt = \frac{1}{\sqrt{2}}\langle-\cos t, -\sin t, 0\rangle$,
$\vec \kappa = d\vec T/ds = \frac{1}{2}\langle-\cos t, -\sin t, 0\rangle$,
$\kappa = |\vec \kappa|= \frac{1}{2}$, so $\rho = 2$,
$\vec N = \vec \kappa /\kappa = \langle-\cos t, -\sin t, 0\rangle$,
the center of curvature is at $\vec r+\rho\vec N = \langle\cos t,\sin t,
t\rangle+2\langle-\cos t, -\sin t, 0\rangle = \langle-\cos t,-\sin t, t\rangle$,
$\vec B = \frac{1}{\sqrt{2}}\langle\sin t,-\cos t, 1\rangle$,
$d\vec B/dt = \frac{1}{\sqrt{2}}\langle\cos t,\sin t, 0\rangle$,
$d\vec B/ds = \frac{1}{2}\langle\cos t,\sin t, 0\rangle$,
$\tau = - \frac{1}{2}\langle\cos t,\sin t, 0\rangle\cdot \langle-\cos t, -\sin t, 0\rangle =
\frac{1}{2}$,
$a_T=0$, $a_N=1$. In general you should not expect to find that $\kappa, \rho,
\tau,a_T, a_N$ are integers, but rather some complex function of $t$. 
Even for parabolas, these formulas get messy really soon.  Do a few
problems from the text to make sure you understand how the
computations proceed, but spend a majority of your time making sure
you can prove the relationships found between the vectors.







%%% Local Variables: 
%%% mode: latex
%%% TeX-master: "../multivariable-calculus"
%%% End: 





