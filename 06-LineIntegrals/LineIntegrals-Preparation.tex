\section{Preparation}

\subsection{Lesson Plans}

This chapter covers the following ideas. When you create your lesson plan, it should contain examples which illustrate these key ideas. Before you take the quiz on this unit, meet with another student out of class and teach each other from the examples on your lesson plan. 

% A list of objectives for the chapter
%\begin{enumerate}
%\item ...
%\end{enumerate}

\begin{enumerate}
\item Describe how to integrate a function along a curve. Use line
  integrals to find the area of a sheet of metal with height
  $z=f(x,y)$ above a curve $\vec r(t)=\langle x(t),y(t)\rangle$ and the average
  value of a function along a curve.
\item Find the following geometric properties of a curve: centroid,
  mass, center of mass, moments of mass, moments of inertia, and radii
  of gyration.
\end{enumerate}

%%% Local Variables: 
%%% mode: latex
%%% TeX-master: "../multivariable-calculus"
%%% End: 
%$


%\subsection{Preparation Problems}

%Here are the preparation problems for this unit.

\subsection{Homework}

In the following list, the ``basic practice'' problems should be quick
problems to help you master the ideas.  The ``good problems'' will
require a little more work.  The theory and application problems are
ones that will challenge you more; make sure you do the problems from
this area to fully master the material.  

{\noindent %\footnotesize 
\begin{tabular}{|l|c|l|l|l|l|}\hline
Topic &Sec &Basic Practice &Good Problems &Thy/App \\\hline
Line Integrals & 15.2&1--18, 21--26, 61--68 & 19--20, 79 & 69--74, 81--82\\\hline
\end{tabular}

}

In addition to these problems, make up your own examples like in the
last section in this chapter.  Compute the quantities we talked about
in this chapter, and use a computer to double-check your answer (Sage
worksheets are provided which calculate all these quantities).

\subsection{Webcasts}

Ben Woodruff has posted some webcasts covering topics in this chapter
and the next: \url{http://www.youtube.com/user/bmwoodruff#g/c/04DF68E73B7ECD54}.

%%% Local Variables: 
%%% mode: latex
%%% TeX-master: "../multivariable-calculus"
%%% End: 
