\section{Preparation}

\noindent
This chapter covers the following ideas. When you create your lesson plan, it should contain examples which illustrate these key ideas. Before you take the quiz on this unit, meet with another student out of class and teach each other from the examples on your lesson plan. 


\begin{enumerate}

\item Model motion in the plane using parametric equations. In particular, how do you describe circles, ellipses, and lines using parametric equations.
\item Be able to convert between rectangular and polar coordinates. 
\item Graph polar functions in the plane. Find intersections of polar equations, and illustrate that not every intersection can be obtained algebraically (you may have to graph the curves).
\item Find derivatives, tangent lines, area, arc length, and surface area using parametric and polar equations.

\end{enumerate}


%%% Local Variables: 
%%% mode: latex
%%% TeX-master: "../multivariable-calculus"
%%% End: 



\subsection{Preparation and Homework Suggestions}

Most class sessions will begin with us presenting prepared material
(``Preparation problems'') in groups.  Typically there will be 4
preparation problems assigned for each day. Each member of the group
should prepare one of these problems before class and teach the rest
of the group what is needed to complete this problem. You will
occasionally select a problem which is entirely new to you, which you
have never seen modeled before. When this occurs, you should look for
examples similar to this problem in the text to learn how to do the
problem. You will grow in skill and in confidence as you study and
prepare these new preparation problems. The new problem will normally
be the last one listed in the preparation problems, so I suggest that
as a group you alternate who takes this problem so that everyone has a
chance to grow this way.


\begin{center}
\begin{tabular}{ll}
&Preparation Problems\\
\hline\hline
Day 1& 10.2:8, 10.2:51, 10.2:66, 10.2:69
\\\hline
Day 2& 10.3:7, 10.3:76, 10.3:99
\\\hline
\end{tabular}
\end{center}

 In the
following list, the ``basic practice'' problems should be quick
problems to help you master the ideas.  The ``good problems'' will
require a little more work.  The theory and application problems are
ones that will challenge you more; make sure you do the problems from
this area to fully master the material.
\medskip
{\noindent \footnotesize 
\begin{tabular}{|l|c|l|l|l|l|}\hline
Topic &Sec &Basic Practice &Good Problems &Thy/App \\\hline
Parametric Equations & 10.2 & 1--32 & 33--37, 39--40, 43--46, 51--62 & 38, 63-66, 67--72 \\\hline
Calculus of Parametric Equations & 10.3 & 1--16, 43--52, 63--66 & 17--42, 53--56, 67--72 & 57--100 \\\hline
Polar coordinates & 10.4 & 1-21, 23--42, 73--80 & 43--53, 59--60 & 54--58, 81--92\\\hline
Area and arclength & 10.5 & 1--26, 45--48 & 27--30, 31--40& 41--44, 69--77\\\hline

\end{tabular}
\smallskip
}

Don't worry about trying to solve by hand all of the integrals for calculating arc length, surface area, etc..  If you can set them up, and solve the simpler ones, you are doing great.


%%% Local Variables: 
%%% mode: latex
%%% TeX-master: "../multivariable-calculus"
%%% End: 
