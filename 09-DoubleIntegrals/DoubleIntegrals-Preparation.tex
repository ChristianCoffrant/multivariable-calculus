\section{Preparation}

\subsection{Lesson Plans}

This chapter covers the following ideas. When you create your lesson plan, it should contain examples which illustrate these key ideas. Before you take the quiz on this unit, meet with another student out of class and teach each other from the examples on your lesson plan. 

% A list of objectives for the chapter
%\begin{enumerate}
%\item ...
%\end{enumerate}

\begin{enumerate}
\item Explain how to setup and compute a double integrals, as well as
how to interchange the bounds of integration. Use these ideas to find
area and volume.
\item For planar regions, find area, mass, centroids, center of mass,
moments of interia, and radii of gyration.
\item Explain how to change coordinate systems in integration, in
particular to a polar coordinates. Explain what the Jacobian of a
transformation is and how to use it.
\item Explain how to use Green's theorem to compute flow along and
flux across a curve.
\end{enumerate}

%%% Local Variables: 
%%% mode: latex
%%% TeX-master: "../multivariable-calculus"
%%% End: 
%$


%\subsection{Preparation Problems}

%Here are the preparation problems for this unit.

\subsection{Homework}

In the following list, the ``basic practice'' problems should be quick
problems to help you master the ideas.  The ``good problems'' will
require a little more work.  The theory and application problems are
ones that will challenge you more; make sure you do the problems from
this area to fully master the material.  

\smallskip 
{\noindent
\begin{tabular}{|l|c|l|l|l|l|}\hline
Topic &Sec &Basic Practice &Good Problems &Thy/App \\\hline
Double integrals & 13.1 & 1--8 & 9--14, 15--22, 25, 26, 29 & 23, 24, 28, 30, 31\\\hline
Double  integrals & 13.2 & 1--24 & 25--32, 36--40 & 33--35, 41, 42\\\hline
Double integrals & 13.3 & 1--20, 33--38 & 21--32, 39-44 & 45, 46\\\hline
Polar coordinates & 13.4 & 1--26 & 27--30 & 31--39\\\hline
Applications & 13.5 & 1--14 & 15--22 & 23--36\\\hline
Changing variables & 13.9 & 1--20 & & 23--25\\\hline
Green's Theorem & 14.4 & 1--12 & 14--16, 20, 23, 24 & 17--19, 21, 22, 25, 26, 27--30\\\hline
\end{tabular}
} 
\medskip

It is crucial that you do not attempt to solve every integral.  For
the most part, you are learning to set up integrals in high
dimensions. You should do many more problems a day if you want
to become proficient at setting up multiple integration problems.  \textbf{I would suggest
  that you do at least 15 or more problems a day in the assigned
  sections}, where you spend time setting up integrals and not solving
them.


%%% Local Variables: 
%%% mode: latex
%%% TeX-master: "../multivariable-calculus"
%%% End: 
